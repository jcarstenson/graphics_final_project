\section{Evaluation}

\par The project was both a success and a failure: we managed to implement a fully functioning object parser that allowed us to render complex objects in our raytracer through the use of .obj files. And while our octree implementation did not work perfectly in the end, we feel it is extraordinarily close to being a complete implementation and we learned a lot from designing it and working through the equations and algorithms necessary to make it efficient and effective. Normally we would have gauged the success by operational speed ups and how much faster the octree could make our ray tracer. Unfortunately, due to the unwieldy and intricate nature of the structure, we found it nearly impossible to debug and discover the issues buried deep within it. Fortunately we at least had the parser as a shiny new feature of our ray tracer and the octree does speed it up considerably, but it does not always render complete images unfortunately.
\par Our octree does appear to work but it has some issues with using small values for maxData and dropping shapes in general. It definitely speeds up rendering by a huge degree, but it commonly drops shapes from the scene which is a pretty glaring issue. If we had to determine the source of our errors in the octree, we believe we could narrow it down to either the traversal algorithm we used or/and the insert function implementation. While debugging we discovered that for some reason we were getting negative t values where we shouldn't have been getting negatives due to the assumptions of our parametric algorithms, but we could not find the source of this as it seemed to be nearly random. As for the insert function we tested it thoroughly and believe it to be working, yet strange problems begin to occur if we set the max number of shapes in a node to lower numbers and we believe this could have caused problems when there were shapes very close together and thus going deeper and deeper down into further octree root nodes. However, after countless hours debugging we decided to leave it alone. So in its current state it renders faster with octrees but they are usually incomplete.
\par The parser is harder to determine the success of as it was very difficult to test since most objects took hours to render so we could only test so many times. We spent a lot of time messing with coordinates trying to make a test scene look good, but this was again difficult as it took hours at a time to render between tries. It would have been great to get the octree working as this would allow us to get much better images from our parser.


%Did you achieve success?  What metrics should we care about and why?  How did
%you evaluate using those metrics, and what were the results?

