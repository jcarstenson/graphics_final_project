\section{Related Work}

% What gets filled in by a \cite{} command comes from the bibliography file.
\par Due to the abstract and complex nature of 3D data structures, we needed to do some research into implementations of both the structure itself and also algorithms for traversal using a ray tracer. A student at the University of Clemson, Brandon Pelfrey has a multitude of free graphics resources, including some related to octrees in particular. His in depth discussion of what an octree is and how it functions helped us to understand the basic ideas behind the implementation and allowed us to formulate our own solution \cite{octree}. Additionally, he implemented his own basic octree that we used as a model for our structure implementation \cite{octreecode}. Unfortunately, his structure is extremely basic when compared with what we wished to do with ours as his octree is coded to store "point" data types and thus the methods relating to node intersections and inserting data were far more simple than what we needed our octree to do as we added entire shapes to our octree and determining whether a bounding box for a shape is intersecting with a node is far more complex than simply determining whether a point is within a node. Despite this, the basic structure of his octree proved extremely useful in creating a skeleton for our implementation, especially concering what data members we needed to create and what methods we needed to implement.
\par While the structure of the tree itself was difficult to visualize, we were able to grasp it with relative ease. This was not the case with traversal algorithms. There are many methods for traversing 3D space in an efficent way, but the main idea for connecting our octree and our ray tracer was to determine which leaf nodes a ray pierces as it goes through the octree. Though seemingly a simple task, upon research we discovered that the most efficient way was to use complex parametric algorithms to compute face intersections. This is all outlined in our most important resource, "An Efficient Parametric Algorithm for Octree Traversal", an article from the Journal of WSCG written by three Spanish researchers in 2000 \cite{traversal}. This paper outlines numerous equations we implemented and connected to our ray tracer in order to calculate face intersections, ray exits, and second node intersections. The combination of all this math and pseudo code allowed us to ultimately create what we believe to be a nearly fully functioning octree, and it would have been impossible without these related works.



%What other projects, papers, ideas are influential to your project~\cite{cudabook, qt5:web}?  These
%could be academic papers, books, open source code, or anything else that's out there.


